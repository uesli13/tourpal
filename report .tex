\documentclass[12pt,a4paper]{article}
\usepackage[utf8]{inputenc}
\usepackage[english]{babel}
\usepackage{geometry}
\usepackage{amsmath}
\usepackage{amsfonts}
\usepackage{amssymb}
\usepackage{graphicx}
\usepackage{fancyhdr}
\usepackage{booktabs}
\usepackage{array}
\usepackage{longtable}
\usepackage{xcolor}
\usepackage{listings}
\usepackage{url}
\usepackage{hyperref}
\usepackage{titlesec}
\usepackage{enumitem}
\usepackage{subcaption}
\usepackage{float}

% Page geometry
\geometry{
    left=2.5cm,
    right=2.5cm,
    top=3cm,
    bottom=3cm
}

% Custom colors
\definecolor{tourpalblue}{RGB}{0,128,128}
\definecolor{tourpalteal}{RGB}{64,224,208}
\definecolor{codecolor}{RGB}{245,245,245}

% Header and footer
\pagestyle{fancy}
\fancyhf{}
\fancyhead[L]{TourPal - Technical Report}
\fancyhead[R]{ICM Project 2024-2025}
\fancyfoot[C]{\thepage}
\renewcommand{\headrulewidth}{0.4pt}

% Custom section formatting
\titleformat{\section}
{\color{tourpalblue}\Large\bfseries}
{\thesection}{1em}{}

\titleformat{\subsection}
{\color{tourpalblue}\large\bfseries}
{\thesubsection}{1em}{}

% Code listings style
\lstset{
    backgroundcolor=\color{codecolor},
    basicstyle=\ttfamily\footnotesize,
    breakatwhitespace=false,
    breaklines=true,
    captionpos=b,
    deletekeywords={...},
    escapeinside={\%*}{*)},
    extendedchars=true,
    frame=single,
    keepspaces=true,
    keywordstyle=\color{blue},
    language=,
    morekeywords={*,...},
    numbers=left,
    numbersep=5pt,
    numberstyle=\tiny\color{gray},
    rulecolor=\color{black},
    showspaces=false,
    showstringspaces=false,
    showtabs=false,
    stepnumber=1,
    stringstyle=\color{red},
    tabsize=2,
    title=\lstname
}

% Hyperref setup
\hypersetup{
    colorlinks=true,
    linkcolor=tourpalblue,
    filecolor=magenta,
    urlcolor=cyan,
    pdftitle={TourPal Technical Report},
    pdfauthor={José Cerqueira, Panos Lekos},
    pdfsubject={ICM Project Technical Documentation},
    pdfkeywords={Flutter, Mobile Development, Firebase, Travel App}
}

\begin{document}

% Title page
\begin{titlepage}
    \centering
    \vspace*{2cm}
    
    {\Huge\bfseries\color{tourpalblue} TourPal}\\[0.5cm]
    {\Large Technical Report}\\[2cm]
    
    {\large\bfseries Course:} Intro to Mobile Development (ICM)\\[0.5cm]
    {\large\bfseries Project:} Travel Guide Platform\\[2cm]
    
    {\large\bfseries Authors:}\\[0.5cm]
    José Cerqueira (76758)\\
    Panos Lekos (128625)\\[2cm]
    
    {\large\bfseries Date:} June 13, 2025\\[2cm]
    
    \vfill
    
\end{titlepage}

% Table of contents
\tableofcontents
\newpage

\section{Overview}

TourPal is a Flutter mobile app that connects travelers with local tour guides through real-time GPS coordination and live tour experiences. The app uses a dual-role system where users can be both travelers and guides, featuring smart booking, real-time tracking, and digital journaling.

Built with Flutter 3.7.2+ and Firebase, the app demonstrates Clean Architecture principles, BLoC state management, and comprehensive real-time features. The main innovation is live tour coordination where guides and travelers can track each other's locations and progress through tour stops in real-time.

\section{Technical Architecture}

\subsection{Architecture Pattern}

The app follows Clean Architecture with three layers:

\begin{lstlisting}
┌─────────────────────────────────────┐
│    Presentation (BLoC + UI)         │
├─────────────────────────────────────┤
│    Domain (Business Logic)          │
├─────────────────────────────────────┤
│    Data (Firebase + Repositories)   │
└─────────────────────────────────────┘
\end{lstlisting}

Each feature is organized as data/domain/presentation modules. The presentation layer uses BLoC for state management, domain contains business logic and repository interfaces, and data implements Firebase integration and repository patterns.

\subsection{State Management}

We use BLoC pattern with flutter\_bloc 8.1.6 for predictable state management. The app also leverages Provider 6.1.1 for dependency injection and GetIt 8.0.0 for service location. Dartz 0.10.1 provides functional programming utilities for error handling.

\section{Technology Stack}

\subsection{Core Technologies}
\begin{table}[H]
\centering
\begin{tabular}{|l|l|l|}
\hline
\textbf{Technology} & \textbf{Version} & \textbf{Purpose} \\
\hline
Flutter & 3.7.2+ & Cross-platform framework \\
Dart & 3.7.2+ & Programming language \\
Firebase Suite & Latest & Backend services \\
Google Maps Flutter & 2.5.0 & Location services \\
\hline
\end{tabular}
\end{table}

\subsection{Key Dependencies}

\textbf{State Management \& Architecture:}
\begin{itemize}
    \item flutter\_bloc 8.1.6 - BLoC pattern implementation
    \item provider 6.1.1 - Dependency injection
    \item get\_it 8.0.0 - Service locator
    \item dartz 0.10.1 - Functional programming utilities
    \item equatable 2.0.5 - Value equality
\end{itemize}

\textbf{Firebase Integration:}
\begin{itemize}
    \item firebase\_core 3.13.1 - Core Firebase functionality
    \item firebase\_auth 5.5.4 - Authentication
    \item cloud\_firestore 5.6.8 - Real-time database
    \item firebase\_storage 12.4.6 - File storage
    \item firebase\_messaging 15.2.6 - Push notifications
    \item firebase\_app\_check 0.3.2+6 - App verification
\end{itemize}

\textbf{Maps \& Location:}
\begin{itemize}
    \item google\_maps\_flutter 2.5.0 - Map integration
    \item geolocator 14.0.1 - GPS location services
    \item geocoding 3.0.0 - Address conversion
    \item flutter\_polyline\_points 2.0.0 - Route drawing
    \item google\_places\_flutter 2.0.9 - Places API
\end{itemize}

\textbf{Media \& Storage:}
\begin{itemize}
    \item image\_picker 1.0.7 - Camera and gallery access
    \item flutter\_image\_compress 2.1.0 - Image optimization
    \item cached\_network\_image 3.4.1 - Image caching
    \item path\_provider 2.1.1 - File system access
\end{itemize}

\section{Database Design and Security}

\subsection{Firestore Collections}

The database has thirteen main collections optimized for real-time tour coordination:
\begin{itemize}
    \item \textbf{users}: User profiles with dual-role support (travelers/guides)
    \item \textbf{guides}: Guide-specific information with availability subcollections
    \item \textbf{tourPlans}: Reusable tour templates with places subcollection
    \item \textbf{tourInstances}: Scheduled tour sessions with participants
    \item \textbf{bookings}: Reservation management and tour lifecycle
    \item \textbf{tourSessions}: Live tour coordination with participant locations subcollection
    \item \textbf{tourProgress}: Tour navigation and progress tracking
    \item \textbf{userLocations}: Real-time location sharing during active tours
    \item \textbf{tourJournals}: Traveler experiences with entries subcollection
    \item \textbf{conversations}: Guide-traveler communication with messages subcollection
    \item \textbf{reviews}: Tour feedback and rating system
\end{itemize}

\subsubsection{Subcollections}

Several collections include subcollections for hierarchical data organization:
\begin{itemize}
    \item \textbf{guides/availableTimes}: Guide availability schedules
    \item \textbf{guides/unavailableSlots}: Guide blocked time periods
    \item \textbf{tourPlans/places}: Tour location details and itineraries
    \item \textbf{tourSessions/participantLocations}: Real-time location data
    \item \textbf{tourJournals/entries}: Individual journal entries with photos
    \item \textbf{conversations/messages}: Chat message history
\end{itemize}

\subsection{Security Rules}

Firestore security implements comprehensive role-based access with helper functions for authentication and ownership verification:

\begin{lstlisting}[caption=Core Security Functions]
function isAuthenticated() {
  return request.auth != null;
}

function isOwner(userId) {
  return request.auth.uid == userId;
}

function isGuide() {
  return isAuthenticated() && 
         exists(/databases/$(database)/documents/users/$(request.auth.uid)) &&
         get(/databases/$(database)/documents/users/$(request.auth.uid)).data.isGuide == true;
}

function isSessionParticipant(sessionData) {
  return isAuthenticated() && 
         (request.auth.uid == sessionData.guideId || 
          request.auth.uid == sessionData.travelerId);
}
\end{lstlisting}

\subsubsection{Access Control Patterns}

The security model implements different access patterns for various use cases:

\textbf{Public Discovery:} Collections like \texttt{users}, \texttt{guides}, \texttt{tourPlans}, and \texttt{reviews} allow public read access to enable tour discovery and guide selection.

\textbf{Participant-Only Access:} \texttt{tourSessions}, \texttt{conversations}, and location tracking collections restrict access to authenticated tour participants.

\textbf{Owner-Controlled:} \texttt{bookings} and \texttt{tourJournals} allow only owners to create content, with flexible read permissions for tour coordination.

\textbf{Draft Support:} Tour creation supports draft status with relaxed validation, allowing iterative tour development before publication.

The security rules balance open discovery with strict privacy controls, enabling collaborative tour experiences while protecting sensitive location and personal data.

\section{Key Features}

\subsection{Real-Time Tour Coordination}

The most innovative feature is live tour tracking with custom map markers. Guides and travelers share GPS locations during active tours, with numbered markers showing visit order.

Guides can advance tour progression and handle group coordination scenarios through real-time Firebase synchronization.

\subsection{Booking System}

Calendar-based booking prevents conflicts through real-time availability checking using table\_calendar 3.0.9. Supports booking up to 90 days ahead with flexible 30-minute time slots. Complex status management handles the complete tour lifecycle from booking to completion.

\subsection{Dual-Role Architecture}

Single accounts support both traveler and guide modes with seamless role switching. Users can explore as travelers while also offering guide services, creating a community-driven platform. The guide registration system includes availability management and specialized UI components.

\section{Security and Privacy}

API keys and sensitive configuration use flutter\_dotenv 5.2.1 with .env files excluded from version control. Multi-layer input validation prevents malicious input. Location tracking uses granular permissions through permission\_handler 12.0.0 with clear user consent.

Firebase provides transport encryption while the app implements secure token storage through shared\_preferences 2.2.2 and session management.

\section{Technical Achievements}

\subsection{Key Accomplishments}
\begin{itemize}
    \item Clean Architecture implementation with feature-based organization
    \item Real-time GPS coordination with conflict resolution
    \item Comprehensive Firebase integration across 6 services
    \item Complex state management for live tour experiences
    \item Custom map marker system with Canvas-based rendering
    \item Smart booking system with availability management
    \item Dual-role user system with seamless switching
\end{itemize}

\subsection{Innovation Areas}

The real-time tour coordination moves beyond static booking platforms to dynamic, collaborative experiences. Custom map markers provide intuitive visual feedback about tour progress. The dual-role system recognizes that users often want both traveler and guide capabilities.

Digital journaling transforms passive tour consumption into active memory creation with location-aware entries and real-time photo sharing.

\section{Project Structure}

\subsection{Directory Organization}

The project follows Flutter's standard structure with feature-based organization for scalability and maintainability:

\begin{verbatim}
tourpal/
├── lib/                          # Main application code
│   ├── app.dart                  # App configuration and theming
│   ├── main.dart                 # Application entry point
│   ├── core/                     # Shared utilities and services
│   ├── features/                 # Feature-based modules
│   │   ├── auth/                 # Authentication system
│   │   ├── bookings/             # Booking management
│   │   ├── dashboard/            # Main application screens
│   │   ├── navigation/           # App navigation system
│   │   ├── profile/              # User profile management
│   │   └── tours/                # Tour management & live coordination
│   ├── models/                   # Shared data models
│   └── repositories/             # Data access abstractions
├── .env                          # Environment variables
├── test/                         # Test files
├── android/                      # Android platform files
├── assets/                       # Static assets
├── firestore.rules               # Database security rules
├── storage.rules                 # Storage security rules
├── pubspec.yaml                  # Dependencies and metadata
└── README.txt                    # Setup instructions
\end{verbatim}

\subsection{Feature Module Structure}

Each feature follows Clean Architecture principles with consistent organization:

\begin{verbatim}
features/feature_name/
├── data/                        # Data layer
│   ├── datasources/             # Remote/local data sources
│   ├── models/                  # Data transfer objects
│   └── repositories/            # Repository implementations
├── domain/                      # Business logic layer
│   ├── repositories/            # Repository interfaces
│   └── usecases/                # Business use cases
└── presentation/                # UI layer
    ├── bloc/                    # BLoC state management
    ├── screens/                 # UI screens
    └── widgets/                 # Feature-specific widgets
\end{verbatim}

This modular approach enables independent development, testing, and maintenance of features while maintaining clear separation between layers and dependencies.

\section{Results and Learning}

TourPal demonstrates modern Flutter development with advanced state management, real-time features, and comprehensive backend integration. The project showcases Clean Architecture benefits, Firebase real-time capabilities, and complex UI coordination.

Key learning areas include real-time application architecture, NoSQL database design, location services integration, and production-quality error handling. The custom map marker system demonstrates advanced Canvas manipulation and bitmap generation.

The project proves ability to integrate multiple complex systems (Firebase, Google Maps, real-time coordination) into a cohesive user experience while maintaining clean architecture principles.

\section{Platform Support}

Currently, TourPal supports Android and Windows platforms. The project structure includes comprehensive Android configuration with proper Firebase integration and Google Services setup. Windows support enables desktop development and testing workflows.

Future expansion could include iOS support with minimal architectural changes due to Flutter's cross-platform nature.

\section{Conclusion}

TourPal successfully combines modern mobile development practices with innovative travel technology. The real-time coordination features create new possibilities for collaborative travel experiences while the technical architecture provides a solid foundation for scaling.

The project demonstrates proficiency in Flutter development, Firebase integration, custom UI rendering, and real-time application design within a clean, maintainable architecture. The comprehensive feature set and robust technical implementation showcase advanced mobile development capabilities.

\end{document}